\documentclass{article}


\usepackage{arxiv}

\usepackage[utf8]{inputenc} % allow utf-8 input
\usepackage[T1]{fontenc}    % use 8-bit T1 fonts
\usepackage{hyperref}       % hyperlinks
\usepackage{url}            % simple URL typesetting
\usepackage{booktabs}       % professional-quality tables
\usepackage{amsfonts}       % blackboard math symbols
\usepackage{nicefrac}       % compact symbols for 1/2, etc.
\usepackage{microtype}      % microtypography
\usepackage{lipsum}
\usepackage{graphicx}

\title{Nested 3D neural networks for kidney and tumor segmentation}


\author{
  Olmo Zavala-RomeroDavid 
  %\thanks{Use footnote for providing further webpage, alternative address)---\emph{not} for acknowledging funding agencies.} \\
  Department of Radiation Oncology\\
  University of Miami, Miller School of Medicine
    1475 NW 12th Avenue
  Miami, FL, PA 33136\\
  \texttt{osz1@miami.edu} \\
  %% examples of more authors
  \And
  Javier \\
  \texttt{} \\
  \AND
  Ignacio\\
  \texttt{} \\
  \AND
  Radka\\
  \texttt{} \\
  \AND
  Adrian\\
  \texttt{} \\
  Jorge\\
  \texttt{} \\
  Rosario\\
  \texttt{} \\
}

\begin{document}
\maketitle

\begin{abstract}
\lipsum[1]
\end{abstract}


% keywords can be removed
\keywords{First keyword \and Second keyword \and More}


\section{Introduction}
\label{sec:intro}
\section{Methodology}
\label{sec:methods}

\subsection{Preprocessing}
\label{sec:prepro}
The proprocessing steps performed are:
\begin{enumerate}
    \item Resample the image to a resolution of $2 \times 2 \times 2$ mm. 
    \item Crop a cube of size $168^3$ from the center of the image. 
    \item Normalize the intensities to a range of 0 to 1.
\end{enumerate}

\subsection{Neural Network Architecture}
\label{sec:nnarc}
The proposed method contains two NN with the same architecture, one is used to segment the kidneys and the other one 
is used to segment the tumors. 

Each NN is a 3D U-Net with 14 convolutional layers. Each convolutional layer uses relu as the activation function and
all the filters are of size of $3\times 3 \times 3$. The input size is the preprocessed image with size $168^3$. 
Last layer is sigmoid.
Uses SGD batchnormalization.
The loss function is negative DSC.
For the tumor the loss function only takes into account the values inside the predicted kindey. 

\begin{figure}[h]
    \centering
    \includegraphics[totalheight=.20\textheight]{imgs/nn.png}
    \caption{NN architecture }
    \label{fig:mobile1}
\end{figure}


\end{document}
